% CONCLUSION CHAPTER
\chapter{Conclusion}\label{ch:conclusion}

\section{Summary of Works}\label{sec:Summary of Works}

This study aimed to investigate the ability to accurately predict the outcome of an esports match of League of Legends through the use of the in-game statistics at various intervals throughout the match.
Whilst a few studies have been completed in academic literature about this subject, this study tested a wider range of machine learning algorithms with data that is updated to match the current state of the game - rather than outdated seven-year-old match data.
A combination of Logistic Regression models and Gradient Boosting Classifiers were applied to League of Legends esports data acquired from the website Oracle's Elixir.
This data was collected from the 2021 League of Legends competitive leagues globally, made up of over 100 individual in-game data points.
The application of the logistic regression model in the 10-minute interval produced an improvement of 22\% over the 52.8\% no information rate benchmark.
This improvement only further increased through the next two models of the 15 and 20 minute interval, with the final gradient boosting classifier achieving a prediction accuracy of 85.99\%. \\

In addition, analysis of the features that play a key roles in a game's outcome were undertaken, calculating the importance of these features and the implications of what they could possibly mean at every given stage of the game.
It was found that in the first 15 minutes of the game, it was clear that individual player  statistics were the most impactful in the decision of the game, with Figure~\ref{fig:FeatureImport10} and~\ref{fig:FeatureImport15} showing a clear support in the ability for a player to keep their deaths to a minimum, whilst still creating a minion killed differential and playing a part in any kills that can occur for their team.
This suggests that the first 15 minutes of the game should be concentrated on generating both a gold and experience lead over the opponent in each position, attempting to grab kills whenever possible.
The model also suggests the fact that the mid-lane tower is the most important tower to take on the map, and should be the secondary focus of a team in these first 15 minutes.
Once the 20-minute mark is met, the model now proposes that the singular feature that is of importance is now the neutral monster \ac{baron}.
Therefore, confirming the objective of this study - that a League of Legends esports match can in-fact be predicted and with a high prediction rate. \\

These results are interesting for both players, coaches and managers within the esports industry, given that these findings highlight that this model outperforms the current rivalling academic literature on this subject~\citep{lee2020predicting, silva2018continuous}.
As esports teams aim for greatness, and to ultimately win both their domestic regional leagues and then qualify to compete for the world championship.
They must ensure they are consistently winning matches and are the best in their respective regions, and they can achieve this through efficient practice.
Using models such as these, teams could alter the feature set and produce predictive systems to optimise and streamline their gameplay to improve their overall win-rates.
These systems could highlight key weaknesses in a teams game-plan, thus helping players practice towards these features of high importance found in these models.

\section{Research Limitations}\label{sec:Research Limitations}

There are a few limitations of this dissertation.
Firstly, given the fact that multicollinearity plays a major part in which data can or should be used in a model of League of Legends.
The explanatory variables being so interconnected inside the game leads to an issue where it becomes hard for machine learning algorithms to properly distinguish the true effect of a singular variable, as the consequences of a single effect in one variable can cause a cascading effect onto other independent variables.
This ultimately can cause massive variance or errors in the predictive model where predictions can be slightly incorrect due to these errors cause inside the feature selection process.
Given this knowledge, the results from any prediction modelling about League of Legends and any game within the MOBA category will inherently suffer due to this fact.
This is due to the core gameplay mechanics revolving around the increasing power of a champion through two key aims - experience and gold.
Despite these key aims not being the core objective to achieve the victory, they are highly influential in achieving victory and thus correlated highly with victory, and this means that a player's intention should be to achieve as much gold and experience as possible whilst also denying their opponents.
Moreover, with almost all the actions a player can take in-game resulting in either gold or experience, all the features within a dataset become highly correlated to both gold and experience metrics.
In order to minimise this both these were removed from the final dataset to reduce the correlation between data as much as possible, but some moderate correlations still remained.


\section{Further Research}\label{sec:Further Research}

For future work, the dataset could be expanded to delve further into pre-game effects such as a further look into the champion select process described in Section~\ref{subsec:champ-select}.
A support tool could be produced for this type of model, using an online web application.
This could use champion specific statistics combined with historical match data in order to dictate the correct champion to be selected at a given phase of the champion selection phase, and could predict the highest likelihood champion pick from the opposition.
This would allow for an initial prediction value before the game begins, maximising a teams potential for winning before the game truly begins.
It could also increase the synergistic effects of a given team composition, drafting champions that inherently work well together using interaction terms between all the possible champion picks.
Another piece of research could be investigating the ability for comebacks in these games, and the tools or objectives that teams from losing positions can use in order to make a comeback for victory.
This would analyse winning teams when they fall a certain threshold percentage of team gold behind, and the key features that enabled them to recover from the deficits incurred.

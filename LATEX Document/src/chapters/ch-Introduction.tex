% INTRODUCTION CHAPTER
\chapter{Introduction}\label{ch:introduction}
\section{Context}\label{sec:Context}
Esports is a form of competition using video games where participants will compete either individually or in a team for a chance at victory.
These competitions attract millions of viewers, with estimates of 532 million spectators by the end of 2022, and this value is expected to grow annually at a value of roughly 8.7\%~\citep{newzoo2022viewers}.
The rapid growth in esports has led to the industry becoming professional, with hundreds of players contracted on full-time contracts competing for prize pools of up to \$40 million~\citep{esportsearnings}.
According to~\citet{newzoo2022viewers} this viewership will help the industry generate over \$1.38 billion in revenue by the end of 2022.
As the esports industry continues to grow, so does the importance on teams to win and remain relevant in the industry.\\

In traditional sports, analytics has become an extremely popular field with teams investing heavily in some form of analytics.
These analytics can be used from evaluating opposing teams, to individual player forecasting and even used to decide signings or team selection~\citep{sarlis2020sports, apostolou2019sports}.
\citet{apostolou2019sports, sarlis2020sports} shows that these analytics can be applied for both teams and individual athletes, giving an accurate estimation of key metrics such as goals scored per season or the expected number of shots attempted in a given match.
This form of sports analytics allow teams to judge both themselves and opposing team performances, calculating the likelihood of shot conversion or the current overall form of a team's performance, giving their team an advantage through prior preparation.
This same methodology could be applied to esports, using these machine learning techniques could highlight specific factors both pre-game and in-game, helping analysts and coaches refine strategies within the game.\\

The ease of data collection coming from each match has led to a rise in esports analytics.
In-depth analysis of matches, teams and pre-game factors become key techniques for teams to gain this advantage over their competitors, with teams being required by their leagues to have at least one dedicated coach and analyst similar to traditional sports teams~\citep{LCSRules}.
These coaches and analysts use predictive analytics to maximise their team's likelihood of winning by altering numerous features related to pre-game and in-game strategies, current \gls{meta} analysis and common patterns of their competition~\citep{kokkinakis2021metagaming}.
However, this analysis is often completed manually by watching key highlights of matches using the analyst's intuition and using rudimentary analysis of in-game factors.\\

If matches can be accurately predicted using machine learning techniques, then analysts can provide new opportunities to optimise player strategies and can lead their teams to better outcomes.
Applying the same findings found in~\citet{gray2012customer}, it can be seen that the overall performance and fan satisfaction with a sports team's performance has a measurable impact on revenue via fan attendance and their media response.
Esports fans also appear to increasingly demand skillful performances especially from players that are deemed as \emph{'superstars'}, with these players being more likely to attract new viewers, thus increasing the economic gain of the market~\citep{mangeloja2019economics, ward2019esport}.
It would then be in the interest of both teams and individual players to maximise their abilities and career longevity using these advanced analytics, so they can fully realise their potential;
especially when the volatility of a players job security results in only the top 10\% of players having lasting, stable careers~\citep{ward2019esport}.\\


\section{League of Legends}\label{sec:League of Legends}
\subsection{General Information}\label{subsec:general-information}
League of Legends is a \ac{MOBA} game developed by Riot Games released in 2009, it is one of most popular esports games in the world with over 180 million monthly players and a peak of 73.8 million concurrent viewers~\citep{riotplayercount, upcomerworld2021}.
A \ac{MOBA} is fusion genre of real-time strategy, role-playing and action games in which two sets of teams will compete in a known arena.
The objective of each game is to defeat the opposition by destroying the enemy's base.
Each player will select and control a unique~\gls{champion} with their own set of distinct abilities, this~\gls{champion} will be selected before the game starts and cannot be changed until the game has ended - this will be covered further in Section~\ref{subsec:champ-select}.
Players can strengthen their~\glspl{champion} by gaining experience and gold, this can be done by slaying enemy~\glspl{minion}, \gls{jungle} monsters, enemy structures or enemy \glspl{champion}.
This gold can be spent in the shop allowing players to purchase items that enhance the attributes of their \gls{champion}, as well as various utility items such as \glspl{ward}.\\

\begin{figure}[h!]
    \centering
    \includegraphics[width=0.5\textwidth]{figures/MOBAMap}
    \caption{Map of League of Legends}
    \label{fig:Lolmap}
\end{figure}

A map of League of Legends can be seen in Figure~\ref{fig:Lolmap}.
There are three lanes, Top, Middle and Bottom, with the \gls{jungle} filling the space between these lanes.
Typically, a player will be assigned to each of these lanes including the jungle, the exception being two players assigned to the bottom lane.
The roles are typically labeled as follows:
\begin{itemize}
    \item Top laner - Starts in the top-lane, often a champion who has larger health and resistance to damage with the ability disrupt enemy players.
    \item Jungler - Roams in the jungle, they help their laners whenever possible often using surprise attacks on enemy laners commonly referred to as a \gls{gank}.
    \item Mid laner - Starts in the mid-lane, often a champion who is a spell-caster that can cause magic damage over a wide area or has high single-target burst damage.
    \item Attack Damage Carry (ADC) - Starts in the bottom-lane, often a long-ranged champion that requires gold in order to deal massive damage towards the latter stages of the game.
    \item Support - Also starts in the bottom-lane, often a champion that can provide aid to the ADC throughout the game via protective abilities and utility such as \glspl{ward}.
\end{itemize}
Each coloured dot represents a \gls{tower} that must be taken in order to reach the enemy \Gls{nexus}.
A river separates the territories between the Blue (Team 1) and the Red team (Team 2) along the dotted black line seen in Figure~\ref{fig:Lolmap}.
In this river you can find \Gls{baron} or \Gls{rift herald} in top-side and the \Gls{dragon} in the bottom-side, they are key objectives that will often be contested.

\subsection{Champion Selection}\label{subsec:champ-select}

Champion Selection plays an important part in every game of League of Legends.
Certain champions have inherent synergies with one another, meaning they are beneficial to be picked with each other.
Likewise, some champions are considered counter matchups when they are good at stopping another champion.
This means that picking a good mixture of \glspl{champion} that are solid synergistically, whilst also ensuring the opponents \glspl{champion} do not counter yours is vital.
These ideas are the fundamentals of champion selection, and they are what professional coaches and analysts attempt to solve each week.
Factors such as player champion experience, the current game balance \gls{patch} or a champion's ability to be flexible across different lanes will change champion select from game to game. \\

As seen in Figure~\ref{fig:draft}, the current draft phase works as follows:
\begin{itemize}
    \item Ban Phase 1 begins with the Blue team, in turn each team bans three champions from the pool.
    \item Pick Phase 1 begins with a singular pick from the Blue side, followed by two picks from the Red side.
     Blue side will get two more picks, followed by a singular pick from Red side for three picks each.
    \item It will then enter Ban Phase 2.
    Here both teams will ban two more champions in turn, with Red side starting.
    \item Pick Phase 2 will begin.
    Here Red side get their fourth champion pick, followed by the final two picks from Blue side and finally Red side pick their final champion.
\end{itemize}

\begin{figure}[h!]
    \centering
    \includegraphics[width=1\textwidth]{figures/DraftPhase}
    \caption{A depiction of the Champion Selection phase}
    \label{fig:draft}
\end{figure}

This champion selection structure leads to clear opportunities for teams to ban out champions that are deemed too strong in Ban Phase 1.
Blue side getting the first pick gives them a chance to pick any champion that is deemed too strong that still remains after Ban Phase 1.
Whilst Red side getting the last pick gives a defined opportunity to pick a counter match-up to any given lane.
These factors can give one side the edge based on the \Gls{patch}, leading to varying strength levels of Blue side vs Red side and can make side selection important.

\section{Structure, Aims, and Objectives}\label{sec:Structure, Aims, and Objectives}

The following section explores an overview of this dissertation: \\

The next chapter, Chapter~\ref{ch:literaturereview}, attempts to review the previous literature in the predictive analytics field, exploring the use in traditional sports for both individual and team performance.
It then moves on to how these predictive models have been used in newer esports studies, with games such as League of Legends and DOTA 2, highlighting key techniques that were used to get a better understanding of key predictors and better performance metrics.
The relation of commercial predictive models were briefly discussed in the betting industry with the popularisation of esports betting, and the industry's need for highly accurate models for their odds-makers. \\

Chapter~\ref{ch:methodology} follows the techniques and tools used for predicting the effect of pre-game choices on the outcome of the match.
Beginning with a brief data description, showing how the data is viewed when it is first accessed as well as the meaning behind many of the features.
The whole data preparation phase is then reviewed, followed by how the data was modelled and the technique that were used were explained. \\

Chapter~\ref{ch:results} then proceeds with a discussion of how the results that were produced were going to be evaluated, and how these performance metrics would ultimately judge the implication of these results.
The results for each time interval model is then presented, with the feature importance of each model being shown.
An interpretation of what these performance metrics mean, and a discussion of why certain features are favoured over others.
The results section is then wrapped up with a general overview of all the results, an implication of what they mean for this study and how they impact the field of predictive analytics inside esports. \\

Finally, chapter~\ref{ch:conclusion} will conclude the dissertation giving a summary of the work completed, potential limitations of the works completed, as well as any further opportunities for research. \\

Having pre-established the landscape of esports and its relationship with analytics, it is clear that refinement in the way that this industry uses its highly available data is needed.
Many academics have predicted the outcomes of matches in esports titles such as ~\citet{ani2019victory} and \citet{lin2016league}.
However, performing these studies, few academics attempted to predict these match results using data from specific time intervals within the game, with most using data from the match result.
Predicting using statistics not only will cause the conclusions made from the studies to be highly skewed towards certain factors, but will also miss out on some decisions made inside the early stages of the game that will cause the swing in momentum for one team.
This study also uses updated esports data, with most studies completed using either amateur player data or massively outdated esports data from between 2015 and 2018.
In contrast to other studies, this study uses a much larger, updated dataset and will concentrate more on the overall effects of the game choices that a team will make during the course of the game, and how teams can apply these choices in order to increase their probability of winning.
Therefore, the research question that will be addressed is as follows:

\begin{quote}  \emph{Can the outcome of a League of Legends match be predicted using machine learning?} \end{quote}
